\documentclass[dvipdfmx,cjk,xcolor=dvipsnames,envcountsect,notheorems,12pt]{beamer}
% * 16:9 のスライドを作るときは、aspectratio=169 を documentclass のオプションに追加する
% * 印刷用の配布資料を作るときは handout を documentclass のオプションに追加する
%   (overlay が全て一つのスライドに出力される)
% 図
% 準正規化定理とは何か復習 聴衆が覚えてない
% 標準化定理
% FLOPSに投稿中
% 変更してないのも証明した 書いた方が良い
% 完全性の証明の解説はゆっくり
% 大まかな方針を解説するのでも良い
% スライドを印刷する時は左から右に 隙間に合わせて直す

\usepackage{pxjahyper}% しおりの文字化け対策 (なくても良い)
\usepackage{amsmath,amssymb,amsfonts,amsthm,ascmac,cases,bm,pifont}
\usepackage{graphicx}
\usepackage[thicklines]{cancel}
\usepackage{url}
\usepackage{newtxmath}
\usepackage{bussproofs}
\usepackage{etex}
\usepackage[all]{xy}
\usepackage{mathtools}

% スライドのテーマ
\usetheme{sumiilab}
% ベースになる色を指定できる
%\usecolortheme[named=Magenta]{structure}
% 数式の文字が細くて見難い時は serif の代わりに bold にしましょう
%\mathversion{bold}

%% ===============================================
%% スライドの表紙および PDF に表示される情報
%% ===============================================

%% 発表会の名前とか(省略可)
% \session{TPP 2017}
%% スライドのタイトル
\title{ストリーム処理DSLについてのサーベイ}
%% 必要ならば、サブタイトルも
\subtitle{インターンシップ報告会}
%% 発表者のお名前
\author{水野 雅之}
%% 発表者の所属([] 内は短い名前)
%\institute[東北大学 住井・松田研]{工学部 電気情報物理工学科\\住井・松田研究室}% 学部生
%\institute[東北大学 住井・松田研]{情報科学研究科 情報基礎科学専攻\\住井・松田研究室}% 院生
%% 発表する日
\date{\today}

%% ===============================================
%% 自動挿入される目次ページの設定(削除しても可)
%% ===============================================

%% section の先頭に自動挿入される目次ページ(削除すると、表示されなくなる)
\AtBeginSection[]{
\begin{frame}
  \frametitle{アウトライン}
  \tableofcontents[sectionstyle=show/shaded,subsectionstyle=show/show/hide]
\end{frame}}
%%% subsection の先頭に自動挿入される目次ページ(削除すると、表示されなくなる)
%\AtBeginSubsection[]{
%\begin{frame}
%  \frametitle{アウトライン}
%  \tableofcontents[sectionstyle=show/shaded,subsectionstyle=show/shaded/hide]
%\end{frame}}
%
%% 現在の section 以外を非表示にする場合は以下のようにする

%% \AtBeginSection[]{
%% \begin{frame}
%%   \frametitle{アウトライン}
%%   \tableofcontents[sectionstyle=show/hide,subsectionstyle=show/show/hide]
%% \end{frame}}
%% \AtBeginSubsection[]{
%% \begin{frame}
%%   \frametitle{アウトライン}
%%   \tableofcontents[sectionstyle=show/hide,subsectionstyle=show/shaded/hide]
%% \end{frame}}

%% ===============================================
%% 定理環境の設定
%% ===============================================

%\setbeamertemplate{theorems}[numbered]% 定理環境に番号を付ける
\theoremstyle{definition}
\newtheorem{definition}{定義}
\newtheorem{axiom}{公理}
\newtheorem{theorem}{定理}
\newtheorem{lemma}{補題}
\newtheorem{corollary}{系}
\newtheorem{proposition}{命題}

%% ===============================================
%% ソースコードの設定
%% ===============================================

\usepackage{listings}
%\usepackage[scale=0.9]{DejaVuSansMono}

\definecolor{DarkGreen}{rgb}{0,0.5,0}
% プログラミング言語と表示するフォント等の設定
\lstset{
  language=Haskell,% プログラミング言語
  basicstyle={\ttfamily\small},% ソースコードのテキストのスタイル
  keywordstyle={\bfseries},% 予約語等のキーワードのスタイル
  commentstyle={},% コメントのスタイル
  stringstyle={},% 文字列のスタイル
  frame=trlb,% ソースコードの枠線の設定 (none だと非表示)
  numbers=none,% 行番号の表示 (left だと左に表示)
  numberstyle={},% 行番号のスタイル
  xleftmargin=5pt,% 左余白
  xrightmargin=5pt,% 右余白
  keepspaces=true,% 空白を表示する
  mathescape,% $ で囲った部分を数式として表示する ($ がソースコード中で使えなくなるので注意)
  % 手動強調表示の設定
  moredelim=[is][\itshape]{@/}{/@},
  moredelim=[is][\color{red}]{@r\{}{\}@},
  moredelim=[is][\color{blue}]{@b\{}{\}@},
  moredelim=[is][\color{DarkGreen}]{@g\{}{\}@},
}

\newcommand{\rightarrowdbl}{\rightarrow\mathrel{\mkern-14mu}\rightarrow}

\newcommand{\xtwoheadrightarrow}[2][]{%
  \xrightarrow[#1]{#2}\mathrel{\mkern-14mu}\rightarrow
}

\newcommand{\LET}[3]{\mathbf{let}~#1=#2~\mathbf{in}~#3}
\newcommand{\EXPANDLET}[1]{#1^\pitchfork}
\newcommand{\SIZE}[2]{\parallel #1 \parallel_{#2}}
\newcommand{\FULLBETA}{\xrightarrow{\beta}}
\newcommand{\CALLBYNEED}{\xrightarrow{\mathrm{need}}}
\newcommand{\CALLBYNEEDI}{\xrightarrow{\mathrm{I}}}
\newcommand{\CALLBYNEEDVCA}{\xrightarrow{\mathrm{VCA}}}
\newcommand{\CALLBYNAME}{\xrightarrow{\mathrm{name}}}
\newcommand{\DEMAND}[2]{\mathbf{needs}(#2, #1)}
\newcommand{\STUCK}[2]{\mathbf{needs}_\mathrm{n}(#2, #1)}
\newcommand{\RTCLOS}[1]{#1^*}
\newcommand{\RTCLOSFULLBETA}{\xtwoheadrightarrow{\beta}}
\newcommand{\RTCLOSCALLBYNEED}{\xtwoheadrightarrow{\mathrm{need}}}
\newcommand{\RTCLOSCALLBYNEEDI}{\xtwoheadrightarrow{\mathrm{I}}}
\newcommand{\RTCLOSCALLBYNEEDVCA}{\xtwoheadrightarrow{\mathrm{VCA}}}
\newcommand{\RTCLOSCALLBYNAME}{\xtwoheadrightarrow{\mathrm{name}}}

%% ===============================================
%% 本文
%% ===============================================
\begin{document}
\frame[plain]{\titlepage}% タイトルページ

\begin{frame}
	\frametitle{モチベーション:IoTデバイスの開発}
	\begin{itemize}
		\item センサーからデータが時事刻々と流入
		\item リアルタイムな処理が必要
		\item 計算機資源に限りがある
	\end{itemize}
	\begin{center}
		\Large $\Downarrow$ \\
		\medskip
		低レイヤも制御できるストリーム処理DSLが欲しい
	\end{center}
\end{frame}

\begin{frame}
	\frametitle{当初の実装:Haiyu.hs}
	\begin{itemize}
		\item Actorモデル
			\begin{itemize}
				\item 各コンビネータは非同期に並列動作
			\end{itemize}
		\item Arrowで組み立てる
	\end{itemize}
	\vfill
	\Large
	しかし
	\begin{itemize}
		\item 想定される処理は恐らくGPU律速
			\begin{itemize}
				\item センシングだけ速くても,処理待ちのデータが溜まるだけ
			\end{itemize}
	\end{itemize}
	$\Rightarrow$各コンビネータは同期しても良いのでは
\end{frame}

\begin{frame}
	\frametitle{埋め込みDSL}
	\Large 既存のホスト言語上にDSLを構築する手法
	\vfill
	長所:
	\begin{itemize}
		\item ホスト言語の機能を流用可能
			\begin{itemize}
				\item 構文解析
				\item 型チェック
			\end{itemize}
	\end{itemize}
	\vfill
	短所:
	\begin{itemize}
		\item ホスト言語の表現力に制約される
			\begin{itemize}
				\item ホスト言語が型安全でなければ,\\型安全なDSLは作りづらい
			\end{itemize}
	\end{itemize}
\end{frame}

\begin{frame}
	\frametitle{Haskellのストリーム処理ライブラリ}
	\begin{itemize}
		\item 代表例:Conduit,Pipes,io-streams
		\item 遅延IOを御するために生まれた
			\begin{itemize}
				\item 例外
				\item リソースの管理
			\end{itemize}
		\item 入出力を含んだコード片も\\部品として組み合わせやすくなる
		\item IO等のモナドを合成する設計
	\end{itemize}
\end{frame}

\begin{frame}
	\frametitle{検討したDSL}
	\begin{itemize}
		\item C言語のコードを出力する,Haskellの埋め込みDSL
			\begin{itemize}
				\item メタプログラミングを意識させない
				\item Parametricity(seq等を除く)
				\item 幽霊型
			\end{itemize}
		\item コード生成を行うモナドだけ実装し,
			ストリーム処理ライブラリで包む
			\begin{itemize}
				\item 好みにより取り替え可能に
			\end{itemize}
	\end{itemize}
\end{frame}

\begin{frame}
	\frametitle{テスト実装(コード生成を行うモナド)}
	\begin{itemize}
		\item とりあえず算術式に限定して実装
		\item データフローグラフを強連結成分分解すればCコードが得られる
	\end{itemize}
\end{frame}

\begin{frame}
	\frametitle{問題点}
	\begin{itemize}
		\item コード生成の都合上,ループや\\シェアリングを明示する必要がある
			\begin{itemize}
				\item ストリーム処理ライブラリと組み合わせる障害に
			\end{itemize}
		\item 幽霊型を入れようとすると一癖ある
	\end{itemize}
	\Large
	\begin{center}
		\textit{``Je n\'ai pas le temps.''} --- \'Evariste Galois
	\end{center}
\end{frame}

\end{document}

